% !TEX TS-program = xelatex
% !TEX encoding = UTF-8
\documentclass[varwidth,border={-1.5cm 1cm 1cm 1cm}]{standalone}
\usepackage{amsmath}
\usepackage[math]{mathspec}
\setmainfont{Linux Libertine O}
\setmathfont(Digits,Latin)[Uppercase=Regular]{Linux Libertine O}
\usepackage{array}
\usepackage{multirow}
\usepackage{booktabs}
\renewcommand{\arraystretch}{1.2}

\begin{document}
\begin{center}
\begin{tabular}{rr>{\centering\arraybackslash}m{2em}>{\centering\arraybackslash}m{2em}@{}}
& & \multicolumn{2}{c}{classification} \\
& & $p$ & \multicolumn{1}{|c}{$\neg p$} \\ \cline{3-4}
\multirow{2}{*}{\rotatebox[origin=c]{90}{stimulus}} & $p$ & \multicolumn{1}{|c}{H} & \multicolumn{1}{|c|}{M} \\ \cline{2-4}
& $\neg p$ & \multicolumn{1}{|c}{F} & \multicolumn{1}{|c|}{C} \\ \cline{3-4}
\end{tabular}
\end{center}

\vspace{\baselineskip}

\addtolength{\jot}{1em}
\begin{align*}
\frac{H}{H + M} &= \text{hit rate, recall, sensitivity} \\
\frac{H}{H + F} &= \text{precision} \\
\frac{F}{F + C} &= \text{false alarm rate} \\
\frac{C}{F + C} &= \text{specificity} \\
\frac{H + C}{H + M + F + C} &= \text{accuracy} \\
\frac{M + F}{H + M + F + C} &= \text{error rate} \\
z\left(\frac{H}{H + M}\right) - z\left(\frac{F}{F + C}\right) &= d^{\thinspace\prime}
\end{align*}
\end{document}
